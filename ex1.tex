% Created 2020-02-26 mié 09:51
% Intended LaTeX compiler: pdflatex
\documentclass[twocolumn]{article}
\usepackage[utf8]{inputenc}
\usepackage[T1]{fontenc}
\usepackage{graphicx}
\usepackage{grffile}
\usepackage{longtable}
\usepackage{wrapfig}
\usepackage{rotating}
\usepackage[normalem]{ulem}
\usepackage{amsmath}
\usepackage{textcomp}
\usepackage{amssymb}
\usepackage{capt-of}
\usepackage{hyperref}
\usepackage[margin=2.5cm]{geometry}
\usepackage[spanish, mexico]{babel}
\usepackage{tikz}
\author{Licenciatura en Matemáticas Aplicadas, UAEH}
\date{24 de febrero de 2020}
\title{Primer Examen de Programación Lineal}
\hypersetup{
 pdfauthor={Licenciatura en Matemáticas Aplicadas, UAEH},
 pdftitle={Primer Examen de Programación Lineal},
 pdfkeywords={},
 pdfsubject={},
 pdfcreator={Emacs 25.2.2 (Org mode 9.3.1)}, 
 pdflang={Spanish}}
\begin{document}

\maketitle
\thispagestyle{empty}

\textbf{NOMBRE:}
\hrulefill

\medskip

\textbf{INSTRUCCIONES:} Hay 6 preguntas en este examen, debes escoger cinco
de ellas, marcando algún modo claro e inequívoco las preguntas
escogidas. Recuerda explicar sin escatimar en detalles las respuestas
a las preguntas. Tienes 1 hora y 50 minutos para resolverlo.

\begin{enumerate}
\item Resuelve el siguiente problema:
\begin{equation*}
\begin{aligned}
\text{Maximizar} \quad & x+3y\\
\text{sujeto a} \quad &
  \begin{aligned}
   3x+y &\leq 6\\
   x,y &\geq  0
  \end{aligned}
\end{aligned}
\end{equation*}

\item Resuelve el siguiente problema 
\begin{equation*}
 \begin{aligned}
\text{Maximizar} \quad & 3x_{1}+x_{2}\\
\text{sujeto a} \quad &
  \begin{aligned}
   x_{1} &\leq 5\\
   x_{2} &\leq 4\\
   x_{1}-x_{2} &\leq 3\\
    x_{1},x_{2} &\geq 0
  \end{aligned}
\end{aligned}
\end{equation*}

\item Resuelve el siguiente problema 
\begin{equation*}
 \begin{aligned}
\text{Maximizar} \quad & 2x_{1}+x_{2}\\
\text{sujeto a} \quad &
  \begin{aligned}
   x_{1}-x_{2} &\leq 2\\
   -2x_{1}+x_{2} &\leq 2\\
   3x_{1}+4x_{2} &\leq 12\\
   x_{1}+x_{2} &\geq 1\\
    x_{1},x_{2} &\geq 0
  \end{aligned}
\end{aligned}
\end{equation*}

\item Escribe el siguiente problema en forma estándar y en forma simplex
(No es necesario resolverlo)
\begin{equation*}
\begin{aligned}
\text{Minimizar} \quad & x_{1}-x_{2}\\
\text{sujeto a} \quad &
  \begin{aligned}
   3x_{1}-x_{2} & \geq 3\\
   x_{1}+x_{2} &\geq -8\\
   x_{1},x_{2} &\geq 0
  \end{aligned}
\end{aligned}
\end{equation*}

\item Una compañía mueblera fabrica tres tipos de libreros: el
\guillemotleft{}intelectual\guillemotright{}, el \guillemotleft{}juvenil\guillemotright{}, y el \guillemotleft{}ejecutivo\guillemotright{}. Cada librero es
elaborado utilizando tres tipos de madera: roble, pino y caoba. El
librero tipo \guillemotleft{}intelectual\guillemotright{} requiere 2 unidades cuadradas de hoja de
roble, 6 de pino y 4 de caoba. El librero tipo \guillemotleft{}juvenil\guillemotright{} requiere
respectivamente 1, 4 y 3 unidades cuadradas de hoja de roble, pino
y caoba. Y el librero tipo \guillemotleft{}ejecutivo\guillemotright{} requiere respectivamente 2,
2 y 8 unidades cuadradas de hoja de pino, roble y caoba.

La ganancia por librero vendido de los tipos \guillemotleft{}intelectual\guillemotright{},
\guillemotleft{}juvenil\guillemotright{} y \guillemotleft{}ejecutivo\guillemotright{} es respectivamente de \$20, \$5 y
\$30. Suponiendo que la compañía dispone en sus bodegas de 100
unidades cuadradas de hoja de roble, 600 unidades de pino y 320
unidades de caoba, plantea el problema de encontrar la producción
que maximice el ingreso como un problema de programación lineal.




Solucion:vamos a usar las variables  $x_1$, $x_2$, $x_3$,

donde $x_1$
representa la cantidad de libreros tipo intelectual.

$x_2$
representa la cantidad de libreros tipo juvenil.

$x_1$
 representa la cantidad de libreros tipo ejecutivo.
 con base a la ganancia en terminos de  $x_1$, $x_2$, $x_3$
 esta dada por:$20x_1+5x_2+30x_3+$
 cada tipo de madera nos da unas restricciones.
 por ejemplo los datos del roble nos dice que $2x_1 +x_2 +2x_3+\leq
 100$

 ,los datos del pino nos dice que, $6x_1+4x_2+2x_3+\leq 600$

 ,los datos del caoba nos dice que, $4x_1+3x_2+8x_3+\leq 100$
 es decir,como un problema de programacion lineal, queda:
 \begin{equation*}
 \begin{aligned}
\text{Maximizar} \quad & 3x_{1}+x_{2}\\
\text{sujeto a} \quad &
  \begin{aligned}
    2x_{1}+x_{2}+2x_{3} &\leq 100\\
    6x_{1}+4x_{2}+2x_{3} &\leq 600\\
    4x_{1}+3x_{2}+8x_{3} &\leq 320\\
    x_{1},x_{2},x_{3} &\geq 0\\

  \end{aligned}
\end{aligned}
\end{equation*}
con base a el programa pyton se deben de producir juvenil,intelectual
y ejecutivo 
\item Supongamos que un estudio ha demostrado que la demanda de autobuses
en cierta ciudad con respecto a la hora del día está regida por la
siguiente tabla:
\begin{center}
\begin{tabular}{lr}
Hora & Cantidad de autobuses\\
\hline
12am-4am & 4\\
4am-8am & 8\\
8am-12pm & 15\\
12pm-4pm & 10\\
4pm-8pm & 17\\
8pm-12am & 5\\
 & \\
\end{tabular}
\end{center}
Supongamos que un autobús debe operar exactamente ocho horas
consecutivas, y que operan en turnos empezando cada cuatro horas a
partir de las 12am. Plantea el problema de encontrar la cantidad
mínima de autobuses que deben adquirirse para cubrir la
demanda como un problema de programación lineal.
\end{enumerate}
\end{document}