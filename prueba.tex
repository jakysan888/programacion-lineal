\documentclass{article}

\usepackage[spanish]{babel}
\usepackage[utf8]{inputenc}

\usepackage{amsmath}


\title{Apuntes de programación lineal}

\author{Jaqueline}


\begin{document}

\maketitle
\tableofcontents
\section{Introducción}
\label{sec:introduccion}


La forma estandar de un problema de programacón lineal es:
Dados una matriz $A$ y vectores $b,c$, maximizar $c^t,x$ sujeto a
$Ax\leq b$
para resolver este tipo de problemas te puedes encontar con los
algunos puntos importantes 
\section{ Puntos importantes }
\label{sec:introduccion}

1:Que en lugar de maximizar se te pida minimizar $f(x)ax+bx$\ , en este caso es más
facil o recomendable multiplicar por un $-1$\ la funcion a minimizar
realizar el procedimiento como si maximizaras y el resultado final
o el valor maximo cambarle el signo y el punto donde se alcanza el maximo es el mismo.


2:puede ser que la función tenga un termino independientes es decir
que sea de la siguiente forma  $f(x)ax+bx+c$, en este caso se resuelve
como si no estuviera el termino $c$ y se resuelve normal per hasta el
ultimo al valor maximo se le debe de agregar el valor de $c$ y el
punto donde se alcanza el máximo permanece igual.




3:




\begin{equation}
  \label{eq:1}
  A=
  \begin{pmatrix}
    0&1&7&6\\
    0&1&7&6

  \end{pmatrix}
  \begin{pmatrix}
     0&1&7&6\\
    0&1&7&6
  \end{pmatrix}
\end{equation}



\begin{tabular}{|c|c|c|}
  \hline 
  &A&B\\
 \hline 
  MAQUINA 1  &1&2\\
  \hline 
  MAQUINA 1  &1&1\\
  \hline 
\end{tabular}
\end{document}c






